
\documentclass{article}
%\usepackage{ulem}     %Used for strikethrough
\usepackage{hyperref} %Used for hyperlink creation
\usepackage{listings}
\usepackage{xcolor}
\lstdefinestyle{customc}{
  belowcaptionskip=1\baselineskip,
  breaklines=true,
  frame=L,
  xleftmargin=\parindent,
  language=C,
  showstringspaces=false,
  basicstyle=\footnotesize\ttfamily,
  keywordstyle=\bfseries\color{green!40!black},
  commentstyle=\itshape\color{purple!40!black},
  identifierstyle=\color{blue},
  stringstyle=\color{orange},
}
\lstset{style=customc}

\begin{document}
\rule{\textwidth}{0.4pt}\textbf{Advertisement :)}

\begin{itemize}
\item \textbf{\href{https://nodeca.github.io/pica/demo/}{pica}} - high quality and fast image
resize in browser.
\item \textbf{\href{https://github.com/nodeca/babelfish/}{babelfish}} - developer friendly
i18n with plurals support and easy syntax.


\end{itemize}
You will like those projects!

\rule{\textwidth}{0.4pt}

\section{h1 Heading 8-)}\subsection{h2 Heading}\subsubsection{h3 Heading}\subsubsection{h4 Heading}



\subsection{Horizontal Rules}

\rule{\textwidth}{0.4pt}

\rule{\textwidth}{0.4pt}

\rule{\textwidth}{0.4pt}



\subsection{Typographic replacements}

Enable typographer option to see result.

(c) (C) (r) (R) (tm) (TM) (p) (P) +-

test.. test... test..... test?..... test!....

!!!!!! ???? ,,  -- ---

"Smartypants, double quotes" and 'single quotes'



\subsection{Emphasis}

\textbf{This is bold text}

\textbf{This is bold text}

\textif{This is italic text}

\textif{This is italic text}

~~Strikethrough~~



\subsection{Blockquotes}



\begin{quote}
Blockquotes can also be nested...\begin{quote}
...by using additional greater-than signs right next to each other...\begin{quote}
...or with spaces between arrows.
\end{quote}
\end{quote}
\end{quote}



\subsection{Lists}

Unordered

\begin{itemize}
\item Create a list by starting a line with \verb|+|, \verb|-|, or \verb|*|
\item Sub-lists are made by indenting 2 spaces:\begin{itemize}
\item Marker character change forces new list start:\begin{itemize}
\item Ac tristique libero volutpat at
\end{itemize}
\begin{itemize}
\item Facilisis in pretium nisl aliquet
\end{itemize}
\begin{itemize}
\item Nulla volutpat aliquam velit
\end{itemize}

\end{itemize}

\item Very easy!


\end{itemize}
Ordered

\begin{enumerate}
\setcounter{enumi}{1}
\item Lorem ipsum dolor sit amet
\item Consectetur adipiscing elit
\item Integer molestie lorem at massa




\item You can use sequential numbers...
\item ...or keep all the numbers as \verb|1.|


\end{enumerate}
Start numbering with offset:

\begin{enumerate}
\setcounter{enumi}{57}
\item foo
\item bar




\end{enumerate}
\subsection{Code}

Inline \verb|code|

Indented code

// Some comments
line 1 of code
line 2 of code
line 3 of code




Block code "fences"

\begin{lstlisting}
Sample text here...

\end{lstlisting}


Syntax highlighting

\begin{lstlisting}[language=js]
var foo = function (bar) {
  return bar++;
};

console.log(foo(5));

\end{lstlisting}


\subsection{Tables}

| Option | Description |
| ------ | ----------- |
| data   | path to data files to supply the data that will be passed into templates. |
| engine | engine to be used for processing templates. Handlebars is the default. |
| ext    | extension to be used for dest files. |

Right aligned columns

| Option | Description |
| ------:| -----------:|
| data   | path to data files to supply the data that will be passed into templates. |
| engine | engine to be used for processing templates. Handlebars is the default. |
| ext    | extension to be used for dest files. |



\subsection{Links}

\href{http://dev.nodeca.com}{link text}

\href{http://nodeca.github.io/pica/demo/}{link with title}

Autoconverted link https://github.com/nodeca/pica (enable linkify to see)



\subsection{Images}

img
img

Like links, Images also have a footnote style syntax

img

With a reference later in the document defining the URL location:





\subsection{Plugins}

The killer feature of \verb|markdown-it| is very effective support of
\href{https://www.npmjs.org/browse/keyword/markdown-it-plugin}{syntax plugins}.



\subsubsection{\href{https://github.com/markdown-it/markdown-it-emoji}{Emojies}}

\begin{quote}
Classic markup: :wink: :crush: :cry: :tear: :laughing: :yum:

Shortcuts (emoticons): :-) :-( 8-) ;)
\end{quote}

see \href{https://github.com/markdown-it/markdown-it-emoji#change-output}{how to change output} with twemoji.



\subsubsection{\href{https://github.com/markdown-it/markdown-it-sub}{Subscript} / \href{https://github.com/markdown-it/markdown-it-sup}{Superscript}}

\begin{itemize}
\item 19^th^
\item H~2~O




\end{itemize}
\subsubsection{\href{https://github.com/markdown-it/markdown-it-ins}{<ins>}}

++Inserted text++



\subsubsection{\href{https://github.com/markdown-it/markdown-it-mark}{<mark>}}

==Marked text==



\subsubsection{\href{https://github.com/markdown-it/markdown-it-footnote}{Footnotes}}

Footnote 1 link[^first].

Footnote 2 link[^second].

Inline footnote^[Text of inline footnote] definition.

Duplicated footnote reference[^second].

[^first]: Footnote \textbf{can have markup}

and multiple paragraphs.


[^second]: Footnote text.



\subsubsection{\href{https://github.com/markdown-it/markdown-it-deflist}{Definition lists}}

Term 1

:   Definition 1
with lazy continuation.

Term 2 with \textif{inline markup}

:   Definition 2

    { some code, part of Definition 2 }

Third paragraph of definition 2.


\textif{Compact style:}

Term 1
~ Definition 1

Term 2
~ Definition 2a
~ Definition 2b



\subsubsection{\href{https://github.com/markdown-it/markdown-it-abbr}{Abbreviations}}

This is HTML abbreviation example.

It converts "HTML", but keep intact partial entries like "xxxHTMLyyy" and so on.

*[HTML]: Hyper Text Markup Language

\subsubsection{\href{https://github.com/markdown-it/markdown-it-container}{Custom containers}}

::: warning
\textif{here be dragons}
:::


\end{document}
      
